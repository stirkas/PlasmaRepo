\documentclass[12pt]{article}

\usepackage{bm}
\usepackage[superscript,biblabel]{cite}
\usepackage{physics} 
\usepackage{siunitx}
\usepackage{esvect}
\usepackage{enumerate}
\usepackage{amsmath}  %I added this so that you can use the align tool for equations!
\usepackage{amssymb}
\usepackage{wasysym}  %This package allows you to put emojis in your paper!!!!

\usepackage{geometry}
 \geometry{
 a4paper,
 total={170mm,257mm},
 left=20mm,
 top=20mm,
 }

\numberwithin{equation}{section}

\begin{document}

\title{A Quasilinear Theory of Anomalous Transport in Axisymmetric Tokamaks}
\author{Stefan Tirkas}


\date{University of Colorado, Boulder\\[2ex]%
      \today}

\maketitle


\section{Introduction}
   \quad Impurity transport is an important issue for fusion plasmas as it can strongly affect the plasma performance.
In order to achieve continuous operation, particle control is essential\cite{ITER}. For D-T burning plasmas, helium ash exhaust is
an important requirement; the helium density as well as impurity ion density must be kept sufficiently low in order to
minimize the dilution of fuel ions. The requirement that the fraction of helium ash remains acceptable ($\lesssim 10\%$) can
only be met if the outward transport of helium is sufficiently rapid\cite{WessonA}. Impurity accumulation also leads to radiative
losses and radiative instabilities, which will further lower confinement times and fusion power output. The neoclassical theory of
impurity transport is highly developed, and it provides a model for the transport of particles, heat, and momentum due to Coulomb
collisions; however, predictions by this theory are rarely well-matched by experiment. A simple representation of particle flux can
be expressed as the sum of a diffusive and convective term,

   \begin{equation}
      \Gamma = -D\frac{dn}{dr} - vn,
   \end{equation}
   %
where $D$ is the diffusivity, and the second term describes an inward pinch with velocity $v$. Diffusivities are generally observed
to be much larger than the neoclassical values predicted, and experiments on JET, for instance, indicate a sharp transition from
approximately neoclassical values of $D$ in the core to very anomalous values in the outer region\cite{WessonB}. Turbulent transport
is considered a plausible candidate for explaining this anomalous transport.

   \quad Perhaps the most important piece of information used here is the particle distribution function - the function that describes the
probability of finding particles at a certain position with a certain velocity in 6-D phase space. The distribution function is taken to be
the standard adiabatic form for now, with plans to update the perturbed distribution function with gyrokinetic effects as well as with trapped
or passing particle dynamics as necessary. We start by assuming a Maxwellian distribution and calculating the species density by integrating over
velocity-space, with $k_B T_j \Rightarrow T_j$,
   
   \begin{equation}
      f_{M,j} = n_j\left(\frac{m_j}{2\pi T_j}\right)^{3/2}e^{-\frac{m_j \epsilon^*_j}{T_j}},
   \end{equation}
   
   \begin{equation}
      \Phi = \delta\Phi
   \end{equation}
   
   \begin{equation}
      n_j = \int_{-\infty}^{\infty}f_M d^{3}v = n_0e^{-\frac{q\delta\Phi}{T_j}} \simeq n_0(1-\frac{q\delta\Phi}{T_j}),
   \end{equation}
   %
with $\epsilon^*_j = \frac{v^2}{2} + \frac{q_j\Phi}{m_j}$, the particle energy per unit mass, and the following assumptions:
the approximation that $\frac{q\delta\Phi}{T_j} \ll 1$, and that of quasi-neutrality, i.e. $n_e = n_i = n_0$. (1.4) makes
clear the following definition for the total distribution function,

   \begin{equation}
      f_j = f_{0,j} + \delta f_j = f_{0,j} - \frac{q_j\delta\Phi}{T_j}f_0,
   \end{equation}
   %
or, more generally as given by Frieman and Chen\cite{FriemanChen},

   \begin{equation}
      \delta f_j = -\frac{q_j}{m_j}\delta\Phi\partial_{\epsilon^*_j}f_{0,j}\;.
   \end{equation}
   
   \quad For the tokamak geometry, we consider, for simplicity, an axisymmetric, large aspect-ratio, circular tokamak.
Using the Grad-Shafranov equation, the following definition for the equilibrium magnetic field can be found,

   \begin{equation}
      \bm{B} = B_\theta\bm{\hat{e}}_\theta + B_\zeta\bm{\hat{e}}_\zeta =
               B_\theta\bm{\hat{e}}_\theta + B_0(1-\epsilon\cos\theta)\bm{\hat{e}}_\zeta =
               B_0[\frac{\epsilon}{q}\bm{\hat{e}}_\theta + (1-\epsilon\cos\theta)\bm{\hat{e}}_\zeta],
   \end{equation}
   %
where $\epsilon=\frac{r}{R_0} \ll 1$ is the inverse aspect ratio, with $R=R_0+r\cos\theta$, for $r$ the minor radius,
and $R_0$ the major radius, and q $\simeq\frac{rB_\zeta}{R_0B_\theta}\sim1$ is the safety factor - the number of
toroidal turns required for one poloidal turn of magnetic field lines. The term $\epsilon\cos\theta$ in $R$ takes into
account the change in toroidal radius along the tokamak midplane. Working to $\mathcal{O}(\epsilon)$, the magnetic field
magnitude, magnetic field unit vector, and toroidal gradient terms can be written as,
   
   \begin{equation}
      B = \sqrt{\bm{B}\cdot\bm{B}} = \sqrt{B_0^2[(1-\epsilon\cos\theta)^2 + (\frac{\epsilon}{q})\strut^2]} =
          B_0\sqrt{1-2\epsilon\cos\theta} \simeq B_0(1-\epsilon\cos\theta),
   \end{equation}

   \begin{equation}
      \bm{\hat{b}} = \frac{\bm{B}}{B} = \frac{\frac{\epsilon}{q}\bm{\hat{e}}_\theta + (1-\epsilon\cos\theta)\bm{\hat{e}}_\zeta}
                     {1-\epsilon\cos\theta} \simeq \frac{\epsilon}{q}(1+\epsilon\cos\theta)\bm{\hat{e}}_\theta + 
                     \bm{\hat{e}}_\zeta \simeq \frac{\epsilon}{q}\bm{\hat{e}}_\theta + \bm{\hat{e}}_\zeta,
   \end{equation}

   \begin{equation}
      \nabla = \partial_r\bm{\hat{e}}_r + \frac{1}{r}\partial_\theta\bm{\hat{e}}_\theta +
               \frac{1}{R}\partial_\zeta\bm{\hat{e}}_\zeta = \partial_r\bm{\hat{e}}_r   +
               \frac{1}{r}\partial_\theta\bm{\hat{e}}_\theta + \frac{1}{R_0 + r\cos\theta}\partial_\zeta\bm{\hat{e}}_\zeta\;.
   \end{equation}


\section{Quasilinear Theory}
   \quad The theoretical picture of turbulent transport is that the free energy released by an instability drives a steady level of
fluctuations in associated perturbed quantities, which results in radial transport of particles and energy. Precise relationships
between the fluctuations and the corresponding transport can be obtained by quasi-linear theory.\cite{WessonC}. In quasi-linear
theory, it is assumed that the plasma is weakly unstable, and that the instability leads to a broad spectrum of waves that modify
the background plasma in a self-consistent way via nonlinear interactions\cite{GurnBhatA}. Generally, quantities of interest are taken to be a main
spatially-averaged part which varies slowly with time compared to the frequency of perturbations, summed with a fluctuation quantity.
Within our plasma description, the density and velocity contributing to the flux are written as,
   
   \begin{equation}
      n = \langle n \rangle + \delta n,
   \end{equation}

   \begin{equation}
      \bm{v} = \delta \bm{v},
   \end{equation}
   %
where $\langle ... \rangle$ represents a flux-surface average, and $\delta \bm{v}$ represents a velocity due to fluctuating quantities.
These two quantities together give a possibility for the anomalous transport, namely,
   
   \begin{equation}
      \Gamma_r = \langle\delta v_r \delta n\rangle,
   \end{equation}
   %
noting that we are considering flow in the radial direction, so averaging over the poloidal angle when considering the tokamak
geometry described in the following section. Also, the lowest order term involves a second-order perturbation because
$\langle \langle n \rangle \delta v_r \rangle = \langle n \rangle \langle \delta v_r \rangle = 0$, since the average of an averaged
quantity is itself, and the average of truly random fluctuations is zero. The density perturbation can be written in terms of a
perturbed distribution function integrated over all velocity space, and the perturbed velocity as an $\bm{E}\times\bm{B}$ drift
where the perturbed $\bm{E}$ field and the background $\bm{B}$ field are used. Taking the poloidal average and bringing the drift
velocity into the velocity-space integral, we get the following result for the quasi-linear particle flux,
   
   \begin{equation}
      \Gamma_r = \langle\int_{-\infty}^{\infty}\delta f \frac{\delta\bm{E}\times\bm{\hat{b}}}{B} d^{3}v\rangle_\theta
               = \frac{1}{2\pi}\int_{0}^{2\pi} \int_{-\infty}^{\infty}\delta f \frac{\delta E_\theta}{B}d\theta d^{3}v \;.
   \end{equation}
   %
Note there is a subtle point that only the $\theta$-component of $\delta E$ remains because we have dropped the term of
$\mathcal{O}(\epsilon)$ in the cross product with (1.9). 


\section{Gyrokinetics}
   \quad $\textbf{Talk about gyrophase-averaging and guiding center coordinates.}$

\subsection{Vlasov Equation}
   \quad The perturbed, gyrokinetic distribution function is given as a combination of adiabatic
and non-adiabatic terms\cite{FriemanChen},

   \begin{equation}
      \delta F = \frac{q}{m}\delta F_a + \delta G,
   \end{equation}
   %
where,
    
   \begin{equation}
      \delta F_a = [\delta\Phi\frac{\partial}{\partial\epsilon^*} + (\delta\Phi - \frac{v_\parallel \delta A_\parallel}
                   {c})\frac{\partial}{B\partial\mu}]F_0, 
   \end{equation}
   
   \begin{equation}
      \delta G_0 = -\frac{q}{m}\langle\delta L\rangle_\alpha\frac{\partial F_0}{B\partial\mu} + \delta H_0,
   \end{equation}

   \begin{equation}
      \langle\ldots\rangle_\alpha = \frac{1}{2\pi}\int_{0}^{2\pi}(\ldots)d\alpha,
   \end{equation}
   %
with $\alpha$ as the gyro-phase angle, $\delta L = \delta\Phi - \frac{\bm{v}\cdot\delta\bm{A}}{c}$, $\epsilon^*$ defined in section 2, and $\mu = \frac{v_\perp^2}{2B}$.
Higher order terms in $\delta G$, the perturbed, non-adiabatic distribution function, are dropped. We can simplify things further by choosing for $f_0$ a Maxwellian
equilibrium distribution function, $f_M$, so that it only depends on $\epsilon^*$ and not $\mu$. This gives us a final distribution function,

   \begin{equation}
      \delta F = \frac{q}{m}\delta\Phi\frac{\partial}{\partial\epsilon^*}f_M + \delta H_0\;.
   \end{equation}
   %
This distribution function can be plugged into the Vlasov equation and gyrophase-averaged to give the standard gyrokinetic Vlasov
equation for a species j\cite{FriemanChen},

   \begin{equation}
   \begin{aligned}
      \partial_t\delta H_0 + v_\parallel\nabla_{X_\parallel}\delta H_0 +
      (\bm{v}_d + \frac{c\bm{\hat{b}}\times\nabla_X\langle\delta\Phi\rangle_\alpha}{B})\cdot\nabla_X\delta H_0 \\
      = 
      -\frac{e_j}{m_j}[\partial_t\langle\delta\Phi\rangle_\alpha\partial_{\epsilon^*} f_M
      -\frac{1}{\omega_{cj}}(\nabla_X\langle\delta\Phi\rangle_\alpha\times\bm{\hat{b}})\cdot\nabla_X f_M],
   \end{aligned}
   \end{equation}
   %
where $\bm{v}_d$, the sum of magnetic curvature and gradient drift terms, is defined as,

   \begin{equation}
   \begin{aligned}
      \bm{v}_d = \frac{v_\parallel^2 + \frac{1}{2} v_\perp^2}{\omega_{cj}}\frac{\bm{B}\times\nabla B}{B^2},
   \end{aligned}
   \end{equation}
   %
with, simplifying to lowest order in $\epsilon$,
   
   \begin{equation}
   \begin{aligned}
      \frac{\bm{B}\times\nabla B}{B^2} &= \frac{B_0[(1-\epsilon\cos\theta)\bm{\hat{e}}_\zeta + \frac{\epsilon}{q}\bm{\hat{e}}_\theta]
      \times(\partial_r\bm{\hat{e}}_r + \frac{1}{r}\partial_\theta\bm{\hat{e}}_\theta + \frac{1}{R}\partial_\zeta\bm{\hat{e}}_\zeta)
      B_0(1-\epsilon\cos\theta)}{B_0^2(1-\epsilon\cos\theta)^2} \\ &=
      \frac{[1-\epsilon\cos\theta)\bm{\hat{e}}_\zeta + \frac{\epsilon}{q}\bm{\hat{e}}_\theta]
      \times[-\frac{1}{R_0}\cos\theta\bm{\hat{e}}_r + \frac{r}{r R_0}\sin\theta\bm{\hat{e}}_\theta]}{(1-\epsilon\cos\theta)^2} \\ &
      \begin{aligned}
         \;= \frac{1}{(1-\epsilon\cos\theta)^2}[&-\frac{(1-\epsilon\cos\theta)\cos\theta}{R_0}(\bm{\hat{e}}_\zeta\times\bm{\hat{e}}_r)
                                               -\frac{(1-\epsilon\cos\theta)\cos\theta}{R_0}(\bm{\hat{e}}_\zeta\times\bm{\hat{e}}_\theta) \\
                                              &-\frac{\epsilon}{q R_0}\cos\theta(\bm{\hat{e}}_\theta\times\bm{\hat{e}}_r)]
      \end{aligned}         
      \\ &\simeq (1+2\epsilon\cos\theta)[-\frac{\cos\theta}{R_0}\bm{\hat{e}}_\theta-\frac{\sin\theta}{R_0}\bm{\hat{e}}_r]
          \simeq -\frac{1}{R_0}(\sin\theta\bm{\hat{e}}_r + \cos\theta\bm{\hat{e}}_\theta)\;.
   \end{aligned}
   \end{equation}
   %
The second and third terms on the left-hand side of (3.6) can be simplified to lowest order in $\epsilon$ using (1.7)-(1.10) and
(3.7)-(3.8) as,
   
   \begin{equation}
   \begin{aligned}
      v_\parallel\nabla_{X_\parallel} \simeq v_\parallel\nabla_\parallel &= 
         v_\parallel(\bm{\hat{b}}\cdot\nabla) = v_\parallel(\frac{\epsilon}{q}\bm{\hat{e}}_\theta +
         \bm{\hat{e}}_\zeta)\cdot(\partial_r\bm{\hat{e}}_r + \frac{1}{r}\partial_\theta\bm{\hat{e}}_\theta + \frac{1}{R_0 +
         r\cos\theta}\partial_\zeta\bm{\hat{e}}_\zeta) \\ &= 
         v_\parallel(\frac{\epsilon}{q r}\partial_\theta + \frac{1}{R}\partial_\zeta) = v_\parallel(\frac{1}{q R_0}\partial_\theta
         + \frac{1}{R}\partial_\zeta) = \frac{v_\parallel}{q R}(\frac{R}{R_0}\partial_\theta + q\partial_\zeta) \\ &=
         \frac{v_\parallel}{q R}((1+\epsilon\cos\theta)\partial_\theta + q\partial_\zeta)
         \simeq \frac{v_\parallel}{q R}(\partial_\theta + q\partial_\zeta) = v_\parallel\frac{\partial}{\partial l},
   \end{aligned}
   \end{equation}

   \begin{equation}
   \begin{aligned}
      \bm{v}_d\cdot\nabla_X \simeq \bm{v}_d\cdot\nabla &= -\frac{v_\parallel^2 + \frac{1}{2}v_\perp^2}{\omega_{cj}}
      (\sin\theta\bm{\hat{e}}_r + \cos\theta\bm{\hat{e}}_\theta)
      \cdot(\partial_r\bm{\hat{e}}_r + \frac{1}{r}\partial_\theta\bm{\hat{e}}_\theta) \\
      &= -\frac{v_\parallel^2 + \frac{1}{2}v_\perp^2}{\omega_{cj}R_0}(\sin\theta\partial_r + \frac{\cos\theta}{r}\partial_\theta),
   \end{aligned}
   \end{equation}
   %
with $l$ being the length along the field lines. Note that we have dropped the non-linear $\bm{E}\times\bm{B}$ drift term on the left-hand
side of (3.6) because we are interested in linearizing this equation.

\subsection{Non-Adiabatic Distribution Function}
   \quad Using a WKB ansatz the gyrophase-averaged terms can be simplified as,
   
   \begin{equation}
      \langle A(\bm{x})\rangle_\alpha = J_0(\sqrt{m_j}\frac{k_\perp v_{\perp j}}{\omega_{cj}}) A(\bm{X}) = J_0(z_j)A(\bm{X})\;.
   \end{equation}
   %
Then, simplifying (3.6) further using (3.9)-(3.11) and taking the Fourier transform, gives,

   \begin{equation}
   \begin{aligned}
      -i\omega\delta\widetilde{H} + v_\parallel\frac{\partial}{\partial l}\delta\widetilde{H} + i \bm{v}_d\cdot\bm{k}_X\delta\widetilde{H} &=
      -\frac{e_j}{m_j}[-i\omega J_0(z_j)\delta\widetilde{\Phi}\partial_{\epsilon^*}f_M \\
      -\frac{i}{\omega_{cj}}J_0(z_j)\delta\widetilde{\Phi}(\bm{k}\times\bm{\hat{b}})\cdot\frac{d}{dr} f_M\bm{\hat{e}}_r] &=
      i\frac{e_j}{m_j}J_0(z_j)\delta\widetilde{\Phi}[\omega\partial_{\epsilon^*} + \frac{k_\theta}{\omega_{cj}}\frac{d}{dr}]f_M,
   \end{aligned}
   \end{equation}

   \begin{equation}
      \Rightarrow\; (v_\parallel\partial_l - i(\omega - \bar{\omega}_{dj}))\delta\widetilde{H} =
      i\frac{e_j}{m_j}J_0(z_j)\delta\widetilde{\Phi}[\omega\partial_{\epsilon^*} + \frac{k_\theta}{\omega_{cj}}\frac{d}{dr}]f_M,
   \end{equation}
   %
with the following definitions,

   \begin{equation}
   \begin{aligned}
      \bar{\omega}_{dj} &= -\bm{v}_d\cdot\bm{k} = \frac{k_\theta(v_\parallel^2 + \frac{1}{2}v_\perp^2)}{\omega_{cj}R_0}(\cos\theta
                           + \frac{k_r}{k_\theta}\sin\theta) \\
                        &= \frac{\omega_{dj}}{2}(\left(\frac{v_\parallel}{v_{Tj}}\right)^2 + \frac{1}{2}\left(\frac{v_\perp}{v_{Tj}}\right)^2)
                           (\cos\theta + \frac{k_r}{k_\theta}\sin\theta),  
   \end{aligned}
   \end{equation}

   \begin{equation}
      \bm{k}\times\bm{\hat{b}} = (k_r\bm{\hat{e}}_r + k_\theta\bm{\hat{e}}_\theta + k_\zeta\bm{\hat{e}}_\zeta)\times
      (\frac{\epsilon}{q}\bm{\hat{e}}_\theta + \bm{\hat{e}}_\zeta) \simeq -k_r\bm{\hat{e}}_\theta + k_\theta\bm{\hat{e}}_r,
   \end{equation}
   %
where $\omega_{dj}$ represents the magnetic curvature drift frequency,
   
   \begin{equation}
      \omega_{dj} = 2\frac{n}{dn/dr}\frac{\omega_{*j}}{R_0},
   \end{equation}
   %
and $\omega_{*j}$ the diamagnetic drift frequency,

   \begin{equation}
      \omega_{*j} = \frac{k_\theta T_j}{q_j B}\frac{1}{n}\frac{dn}{dr}\;.
   \end{equation}
   %
Note that $f_M$ depends only on r due to it being a function of $n_j(r)$ and $T_j(r)$ which are only changing across the circular
flux-surfaces - therefore only functions of radius - and that $\nabla_X \simeq \nabla$ to lowest order for the perturbed distribution
functions and potential. Next, we can simplify further by plugging in the definition of the Maxwellian, giving the following values for
derivatives,

   \begin{equation}
      \partial_{\epsilon^*}f_M = -\frac{m_j}{T_j}n_j\left(\frac{m_j}{2\pi T_j}\right)^{3/2}e^{-\frac{m_j \epsilon^*}{T_j(r)}} =
                                 -\frac{m_j}{T_j}f_M,
   \end{equation}
   
   \begin{equation}
   \begin{aligned}
      \frac{d}{dr}f_M &= \frac{dn}{dr}\frac{d}{dn}f_M + \frac{dT}{dr}\frac{d}{dT}f_M = 
                         \frac{dn}{dr}e^{-\frac{m_j \epsilon^*}{T_j(r)}} + 
                         \frac{dT}{dr}\frac{du}{dT}\frac{d}{du}[n_j\left(\frac{m_j u(T(r))}{2\pi}\right)^{3/2}e^{-m_j \epsilon^* u(T_j(r))}] \\
                      &= \frac{1}{n}\frac{dn}{dr}f_M + \frac{dT}{dr}\frac{du}{dT}n_j[\frac{3}{2}\left(\frac{m_ju}{2\pi}\right)^{1/2}
                         \left(\frac{m_j}{2\pi}\right) - m_j\epsilon^*\left(\frac{m_ju}{2\pi}\right)^{3/2}]e^{-m_j \epsilon^* u(T_j(r))} \\
                      &= \frac{1}{n}\frac{dn}{dr}f_M + \frac{dT}{dr}\frac{du}{dT}[\frac{3}{2}u^{-1} - m_j\epsilon^*]f_M
                       = [\frac{1}{n}\frac{dn}{dr} + \frac{dT}{dr}(-\frac{1}{T^2})(\frac{3}{2}u^{-1} - m_j\epsilon^*)]f_M \\
                      &= [\frac{1}{n}\frac{dn}{dr} - \frac{1}{T}\frac{dT}{dr}(\frac{3}{2} - \frac{m_jv^2}{2T_j})]f_M
                       = [\frac{1}{n}\frac{dn}{dr} - \frac{1}{T}\frac{dT}{dr}(\frac{3}{2} - \frac{v^2}{2v_{Tj}^2})]f_M \\
                      &= \frac{1}{n}\frac{dn}{dr}[1 + (\frac{1}{2}\left(\frac{v}{v_{Tj}}\right)^2 - \frac{3}{2})\eta_j]f_M,
   \end{aligned}
   \end{equation}
   %
given that $u = T^{-1}$, $du = -T^{-2}dT$, $v_{Tj} = \sqrt{\frac{T_j}{m_j}}$, and $\eta_j = \frac{n}{T}\frac{dT}{dn}$. Putting these
derivatives into (3.13) then gives,
   
   \begin{equation}
   \begin{aligned}
      &(v_\parallel\partial_l - i(\omega - \bar{\omega}_{dj}))\delta\widetilde{H} = i\frac{e_j}{m_j}J_0(z_j)\delta\widetilde{\Phi}
      [-\frac{m_j}{T_j}\omega + \frac{k_\theta}{\omega_{cj}}\frac{1}{n}\frac{dn}{dr}[1 + (\frac{1}{2}\left(\frac{v}{v_{Tj}}\right)^2
                                                                                     - \frac{3}{2})\eta_j]]f_M \\
      \Rightarrow\; &(v_\parallel\partial_l - i(\omega - \bar{\omega}_{dj}))\delta\widetilde{H} = -i\frac{e_j}{T_j}J_0(z_j)\delta\widetilde{\Phi}
      [\omega - \frac{k_\theta T_j m_j}{q_j B m_j}\frac{1}{n}\frac{dn}{dr}[1 + (\frac{1}{2}\left(\frac{v}{v_{Tj}}\right)^2
                                                                                     - \frac{3}{2})\eta_j]]f_M \\
      \Rightarrow\; &(-v_\parallel\partial_l + i(\omega - \bar{\omega}_{dj}))\delta\widetilde{H} = i\frac{e_j}{T_j}J_0(z_j)\delta\widetilde{\Phi}
      [\omega - \omega_{*j}[1 + (\frac{1}{2}\left(\frac{v}{v_{Tj}}\right)^2- \frac{3}{2})\eta_j]]f_M
   \end{aligned}
   \end{equation}

   \begin{equation}
      \Rightarrow\; (-v_\parallel\partial_l + i(\omega - \bar{\omega}_{dj}))\delta\widetilde{H}
      = i\frac{e_j}{T_j}J_0(z_j)\delta\widetilde{\Phi}[\omega - \omega_{*j}^T]f_M,
   \end{equation}
   %
with the following definition,
   \begin{equation}
      \omega_{*j}^T = \omega_{*j}[1 + (\frac{1}{2}\left(\frac{v}{v_{Tj}}\right)^2-\frac{3}{2})\eta_j]\;.
   \end{equation}
   %
Then (3.21) can be simplified further and solved for $\delta\widetilde{H}$ if a few physical assumptions are made for electron drift and ITG modes.
First off, trapped particle effects are neglected, i.e.,
   \begin{equation}
      \omega \gg \omega_{bj},
   \end{equation}
   %
for $\omega_{bj}$ as the bounce frequency. We also assume electrons move rapidly in response to the electrostatic potential,
   \begin{equation}
      k_\parallel v_{Te} \gg \omega \gg k_\parallel v_{Ti},
   \end{equation}
   %
noting that $v_{Tj} \simeq v_{\parallel j}$. Finally we assume that the frequency associated with magnetic drifts of ions is much smaller than
that of the perturbed modes,
   \begin{equation}
      \omega \gg \omega_{di}\;.
   \end{equation}
   %
Now, (3.24) and (3.25) allow the left-hand side of (3.21) to be rewritten for ions as,
   \begin{equation}
      i(\omega-\bar{\omega}_{di})(1+i\frac{v_\parallel}{\omega-\omega_{di}}\frac{\partial}{\partial l}) \simeq
      i(\omega-\bar{\omega}_{di})(1-\frac{k_\parallel v_{\parallel i}}{\omega}) \simeq i(\omega-\bar{\omega}_{di})\;.
   \end{equation}
   %
Finally we can solve for $\delta\widetilde{H}$ explicitly by replacing the left-hand side of (3.21) with (3.26), giving
the perturbed, non-adiabatic distribution function for ions,
   \begin{equation}
      \delta\widetilde{H} = \frac{e_j}{T_j}J_0(z_j)\delta\widetilde{\Phi}f_M\frac{\omega-\omega_{*i}^T}{\omega-\bar{\omega}_{di}}\;.
   \end{equation}


\begin{thebibliography}{0}
   
   \bibitem{ITER} ITER Physics Expert Groups on Confinement and Transport and Confine-ment Modelling and Database, Nucl. Fusion \textbf{39}, 2175 (1999).
   \bibitem{WessonA} Wesson J., \& Campbell D. J., \textit{Tokamaks} (Oxford: Clarendon Press, 2004), p.219
   \bibitem{WessonB} Wesson J., \& Campbell D. J., \textit{Tokamaks} (Oxford: Clarendon Press, 2004), p.196-197
   \bibitem{FriemanChen} Frieman E. A., Chen L., The Physics of Fluids \textbf{25}, 502 (1982).
   \bibitem{WessonC} Wesson J., \& Campbell D. J., \textit{Tokamaks} (Oxford: Clarendon Press, 2004), p.202-210
   \bibitem{GurnBhatA} Gurnett, D., \& Bhattacharjee, A., \textit{Introduction to Plasma Physics: With Space and Laboratory Applications}, (Cambridge: Cambridge University Press 2005), p.428
   \bibitem{WessonD} Wesson J., \& Campbell D. J., \textit{Tokamaks} (Oxford: Clarendon Press, 2004), p.112
	
\end{thebibliography}
    
\end{document}