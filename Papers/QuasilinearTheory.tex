\documentclass[12pt]{article}

\usepackage{bm}
\usepackage[superscript,biblabel]{cite}
\usepackage{physics} 
\usepackage{siunitx}
\usepackage{esvect}
\usepackage{enumerate}
\usepackage{amsmath}  %I added this so that you can use the align tool for equations!
\usepackage{wasysym}  %This package allows you to put emojis in your paper!!!!
	
\usepackage{geometry}
 \geometry{
 a4paper,
 total={170mm,257mm},
 left=20mm,
 top=20mm,
 }

\begin{document}

\title{Quasilinear Theory of Anomalous Transport in Axisymmetric Tokamaks}
\author{Stefan Tirkas}


\date{University of Colorado, Boulder\\[2ex]%
      \today}

\maketitle


\section{Introduction}


\section{Tokamak Geometry}
   \quad For simplicity, we consider an axisymmetric, large aspect-ratio, circular tokamak. This gives
the following definition for the equilibrium magnetic field,

   \begin{equation}
      \bm{B} = B_\theta\bm{\hat{e}}_\theta + B_\zeta\bm{\hat{e}}_\zeta =
               B_\theta\bm{\hat{e}}_\theta + B_0(1-\epsilon\cos\theta)\bm{\hat{e}}_\zeta =
               B_0[\frac{\epsilon}{q}\bm{\hat{e}}_\theta + (1-\epsilon\cos\theta)\bm{\hat{e}}_\zeta],
   \end{equation}
   % for no indent next line.
where $\epsilon=\frac{r}{R_0} \ll 1$ is the inverse aspect ratio, with $R=R_0+r\cos\theta$, for $r$ the minor radius,
and $R_0$ the major radius, and q $\simeq\frac{rB_\zeta}{R_0B_\theta}\sim1$ is the safety factor\cite{Wesson} - the number of
toroidal turns required for one poloidal turn of magnetic field lines. The term $\epsilon\cos\theta$ in $R$ takes into
account the change in toroidal radius along the tokamak midplane. Working to $\mathcal{O}(\epsilon)$, the magnetic field
magnitude, magnetic field unit vector, and toroidal gradient terms can be written as,
   
   \begin{equation}
      B = \sqrt{\bm{B}\cdot\bm{B}} = \sqrt{B_0^2[(1-\epsilon\cos\theta)^2 + (\frac{\epsilon}{q})\strut^2]} =
          B_0\sqrt{1-2\epsilon\cos\theta} \simeq B_0(1-\epsilon\cos\theta),
   \end{equation}

   \begin{equation}
      \bm{\hat{b}} = \frac{\bm{B}}{B} = \frac{\frac{\epsilon}{q}\bm{\hat{e}}_\theta + (1-\epsilon\cos\theta)\bm{\hat{e}}_\zeta}
                     {1-\epsilon\cos\theta} \simeq \frac{\epsilon}{q}(1+\epsilon\cos\theta)\bm{\hat{e}}_\theta + 
                     \bm{\hat{e}}_\zeta \simeq \frac{\epsilon}{q}\bm{\hat{e}}_\theta + \bm{\hat{e}}_\zeta,
   \end{equation}

   \begin{equation}
      \nabla = \partial_r\bm{\hat{e}}_r + \frac{1}{r}\partial_\theta\bm{\hat{e}}_\theta +
               \frac{1}{R}\partial_\zeta\bm{\hat{e}}_\zeta = \partial_r\bm{\hat{e}}_r   +
               \frac{1}{r}\partial_\theta\bm{\hat{e}}_\theta + \frac{1}{R_0 + r\cos\theta}\partial_\zeta\bm{\hat{e}}_\zeta\;.
   \end{equation}

\section{Gyrokinetics}
   \quad $\bm{Talk about gyrophase-averaging and guiding center coordinates.}$

\subsection{Vlasov Equation}
   \quad The perturbed, gyrokinetic distribution function is given as a combination of adiabatic
and non-adiabatic terms\cite{FriemanChen},

   \begin{equation}
      \delta F = \frac{q}{m}\delta F_a + \delta G,
   \end{equation}
   %
where,
    
   \begin{equation}
      \delta F_a = [\delta\Phi\frac{\partial}{\partial\epsilon^*} + (\delta\Phi - \frac{v_\parallel \delta A_\parallel}
                   {c})\frac{\partial}{B\partial\mu}]F_0, 
   \end{equation}
   
   \begin{equation}
      \delta G_0 = -\frac{q}{m}\langle\delta L\rangle_\alpha\frac{\partial}{B\partial\mu} + \delta H_0,
   \end{equation}

   \begin{equation}
      \langle\ldots\rangle_\alpha = \frac{1}{2\pi}\int_{0}^{2\pi}(\ldots)d\alpha,
   \end{equation}
   %
with $\alpha$ as the gyro-phase angle, $\delta L = \delta\Phi - \frac{\bm{v}\cdot\delta\bm{A}}{c}$, $\epsilon^* = \frac{v^2}{2} + 
\frac{q\Phi_0}{m}$, and $\mu = \frac{v_\perp^2}{2B}$. Higher order terms in $\delta G$, the perturbed, non-adiabatic distribution function,
are dropped. We can simplify things further by choosing for $F_0$ a Maxwellian equilibrium distribution function, $f_M$, so that it only depends
on $\epsilon^*$ and not $\mu$. This gives us a final distribution function,

   \begin{equation}
      \delta F = \frac{q}{m}\delta\Phi\frac{\partial}{\partial\epsilon^*}f_M + \delta H_0\;.
   \end{equation}
   %
This distribution function can be plugged into the Vlasov equation and gyrophase-averaged to give the standard gyrokinetic Vlasov
equation for a species j\cite{FriemanChen},

   \begin{equation}
   \begin{aligned}
      \partial_t\delta H_0 + v_\parallel\nabla_\parallel\delta H_0 +
      (\bm{v}_d + \frac{c\bm{\hat{b}}\times\nabla_X\langle\delta\Phi\rangle_\alpha}{B})\cdot\nabla_X\delta H_0 \\
      = 
      -\frac{e_j}{m_j}[\partial_t\langle\delta\Phi\rangle_\alpha\partial_{\epsilon^*} f_M
      -\frac{1}{\omega_{cj}}(\nabla_X\langle\delta\Phi\rangle_\alpha\times\bm{\hat{b}})\cdot\nabla_X f_M],
   \end{aligned}
   \end{equation}
   %
where $\bm{v}_d$, the sum of magnetic curvature and gradient drift terms, is defined as,

   \begin{equation}
   \begin{aligned}
      \bm{v}_d = \frac{v_\parallel^2 + \frac{1}{2} v_\perp^2}{\omega_{cj}}\frac{\bm{B}\times\nabla B}{B^2},
   \end{aligned}
   \end{equation}
   %
with, simplifying to lowest order in $\epsilon$,
   
   \begin{equation}
   \begin{aligned}
      \frac{\bm{B}\times\nabla B}{B^2} &= \frac{B_0[(1-\epsilon\cos\theta)\bm{\hat{e}}_\zeta + \frac{\epsilon}{q}\bm{\hat{e}}_\theta]
      \times(\partial_r\bm{\hat{e}}_r + \frac{1}{r}\partial_\theta\bm{\hat{e}}_\theta + \frac{1}{R}\partial_\zeta\bm{\hat{e}}_\zeta)
      B_0(1-\epsilon\cos\theta)}{B_0^2(1-\epsilon\cos\theta)^2} \\ &=
      \frac{[1-\epsilon\cos\theta)\bm{\hat{e}}_\zeta + \frac{\epsilon}{q}\bm{\hat{e}}_\theta]
      \times[-\frac{1}{R_0}\cos\theta\bm{\hat{e}}_r + \frac{r}{r R_0}\sin\theta\bm{\hat{e}}_\theta]}{(1-\epsilon\cos\theta)^2} \\ &
      \begin{aligned}
         \;= \frac{1}{(1-\epsilon\cos\theta)^2}[&-\frac{(1-\epsilon\cos\theta)\cos\theta}{R_0}(\bm{\hat{e}}_\zeta\times\bm{\hat{e}}_r)
                                               -\frac{(1-\epsilon\cos\theta)\cos\theta}{R_0}(\bm{\hat{e}}_\zeta\times\bm{\hat{e}}_\theta) \\
                                              &-\frac{\epsilon}{q R_0}\cos\theta(\bm{\hat{e}}_\theta\times\bm{\hat{e}}_r)]
      \end{aligned}         
      \\ &\simeq (1+2\epsilon\cos\theta)[-\frac{\cos\theta}{R_0}\bm{\hat{e}}_\theta-\frac{\sin\theta}{R_0}\bm{\hat{e}}_r]
          \simeq -\frac{1}{R_0}(\sin\theta\bm{\hat{e}}_r + \cos\theta\bm{\hat{e}}_\theta)          
   \end{aligned}
   \end{equation}
   %
The second and third terms on the left-hand side of (10) can be simplified to lowest order in $\epsilon$ using (1)-(4) and (11)-(12) as,
   
   \begin{equation}
   \begin{aligned}
      v_\parallel\nabla_\parallel &= 
         v_\parallel(\bm{\hat{b}}\cdot\nabla) = v_\parallel(\frac{\epsilon}{q}\bm{\hat{e}}_\theta +
         \bm{\hat{e}}_\zeta)\cdot(\partial_r\bm{\hat{e}}_r + \frac{1}{r}\partial_\theta\bm{\hat{e}}_\theta + \frac{1}{R_0 +
         r\cos\theta}\partial_\zeta\bm{\hat{e}}_\zeta) \\ &= 
         v_\parallel(\frac{\epsilon}{q r}\partial_\theta + \frac{1}{R}\partial_\zeta) = v_\parallel(\frac{1}{q R_0}\partial_\theta
         + \frac{1}{R}\partial_\zeta) = \frac{v_\parallel}{q R}(\frac{R}{R_0}\partial_\theta + q\partial_\zeta) \\ &=
         \frac{v_\parallel}{q R}((1+\epsilon\cos\theta)\partial_\theta + q\partial_\zeta)
         \simeq \frac{v_\parallel}{q R}(\partial_\theta + q\partial_\zeta),
   \end{aligned}
   \end{equation}

   \begin{equation}
   \begin{aligned}
      \bm{v}_d\cdot\nabla_X &= -\frac{v_\parallel^2 + \frac{1}{2}v_\perp^2}{\omega_{cj}}(\sin\theta\bm{\hat{e}}_r + \cos\theta\bm{\hat{e}}_\theta)
                              \cdot(\partial_r\bm{\hat{e}}_r + \frac{1}{r}\partial_\theta\bm{\hat{e}}_\theta) \\
                            &= -\frac{v_\parallel^2 + \frac{1}{2}v_\perp^2}{\omega_{cj}}(\sin\theta\partial_r + \frac{\cos\theta}{r}\partial_\theta)\;.
   \end{aligned}
   \end{equation}
   %
Note that we have dropped the non-linear $\bm{E}\times\bm{B}$ drift term on the left-hand side of (10) because we are
interested in linearizing this equation.

\subsection{Non-Adiabatic Distribution Function}
   \quad Using a WKB ansatz the gyrophase-averaged terms can be simplified as,
   \begin{equation}
      \langle A(\bm{x})\rangle_\alpha = J_0(\frac{k_\perp v_\perp}{\omega_{cj}}) A(\bm{X})
   \end{equation}

\begin{thebibliography}{0}
	
   \bibitem{Wesson} Wesson 2004
   \bibitem{FriemanChen} Frieman, Chen 1982
	
\end{thebibliography}
    
\end{document}