\documentclass[12pt]{article}

\usepackage{bm}
\usepackage[superscript,biblabel]{cite}
\usepackage{physics} 
\usepackage{siunitx}
\usepackage{esvect}
\usepackage{enumerate}
\usepackage{amsmath}  %I added this so that you can use the align tool for equations!
\usepackage{amssymb}
\usepackage{wasysym}  %This package allows you to put emojis in your paper!!!!

\usepackage{geometry}
 \geometry{
 a4paper,
 total={170mm,257mm},
 left=20mm,
 top=20mm,
 }

\numberwithin{equation}{subsection}

\begin{document}

\title{A Quasi-linear Theory of Impurity Transport in Circular, Axisymmetric Tokamaks}
\author{Stefan Tirkas}


\date{University of Colorado, Boulder\\[2ex]%
      \today}

\maketitle


\section{Introduction}

\subsection{Impurity Transport}
   \quad Impurity transport is an important issue for fusion plasmas as it can strongly affect the plasma performance.
In order to achieve continuous operation, particle control is essential\cite{ITER}. For D-T burning plasmas, helium ash exhaust is
an important requirement; the helium density as well as impurity ion density must be kept sufficiently low in order to
minimize the dilution of fuel ions. The requirement that the fraction of helium ash remains acceptable ($\lesssim 10\%$) can
only be met if the outward transport of helium is sufficiently rapid\cite{WessonA}. Impurity accumulation also leads to radiative
losses and radiative instabilities, which will further lower confinement times and fusion power output. The neoclassical theory of
impurity transport is highly developed, and it provides a model for the transport of particles, heat, and momentum due to Coulomb
collisions; however, predictions by this theory are rarely well-matched by experiment. A simple representation of particle flux can
be expressed as the sum of a diffusive and convective term,
   \begin{equation}
      \Gamma = -D\frac{dn}{dr} - vn,
   \end{equation}
   %
where $D$ is the diffusivity, and the second term describes an inward pinch with velocity $v$. Diffusivities are generally observed
to be much larger than the neoclassical values predicted, and experiments on JET, for instance, indicate a sharp transition from
approximately neoclassical values of $D$ in the core to very anomalous values in the outer region\cite{WessonB}. Turbulent transport
is considered a plausible candidate for explaining this anomalous transport.

\subsection{Distribution Function}
   \quad Perhaps the most important piece of information used here is the particle distribution function - the function that describes the
probability of finding particles at a certain position with a certain velocity in 6-D phase-space. The distribution function is taken to be
the standard adiabatic form for now, with plans to update the perturbed distribution function with gyrokinetic effects as well as with trapped
or passing particle dynamics as necessary. We start by assuming a Maxwellian distribution modified by a potential term and with $T_j$ represented
in terms of energy, i.e. $k_B T_j \Rightarrow T_j$,
   \begin{equation}
      f_{0,j} = n_j\left(\frac{m_j}{2\pi T_j}\right)^{3/2}e^{-\frac{m_j \epsilon^*_j}{T_j}}
              = f_{M,j}e^{-\frac{q_j\delta\Phi}{T_j}} \simeq f_{M,j}(1-\frac{q_j\delta\Phi}{T_j}),
   \end{equation}
   %
   \begin{equation}
      \Phi = \delta\Phi
   \end{equation}
   %
with $\epsilon^*_j \equiv \frac{v^2}{2} + \frac{q_j\Phi}{m_j}$, the particle energy per unit mass, and the approximation that
that $\frac{q\delta\Phi}{T_j} \ll 1$. (1.2.1) makes clear the following definition for the total distribution function in the
case of a Maxwellian,
   \begin{equation}
      f_j = f_{0,j} + \delta f_j = f_{M,j} - \frac{q_j\delta\Phi}{T_j}f_{M,j},
   \end{equation}
   %
or, more generally as given by Frieman and Chen\cite{FriemanChen},
   \begin{equation}
      \delta f_j = -\frac{q_j}{m_j}\delta\Phi\partial_{\epsilon^*}f_{0,j}\;.
   \end{equation}
   %
We also assume a condition for quasi-neutrality, which states that as a whole the plasma remains charge-neutral,
   \begin{equation}
      \sum\limits_{j}q_j n_j = 0,
   \end{equation}
   %
where the density $n_j$ is defined as the integration of the distribution function over velocity space,
   \begin{equation}
      n_j = \int f_j d^{3}v\;.
   \end{equation}
   %
Liouville's theorem gives, using Hamiltonian theory, the time-evolution of the phase-space distribution function of a species, namely\cite{WessonD},
   \begin{equation}
      \frac{df_j(\bm{x},\bm{p},t)}{dt} = \frac{\partial f_j}{\partial t}
                                       + \frac{\partial f_j}{\partial \bm{x}}\frac{\partial \bm{x}}{\partial t}
                                       + \frac{\partial f_j}{\partial \bm{p}}\frac{\partial \bm{p}}{\partial t} = 0.                     
   \end{equation}
   %
(1.2.7) can be used to get the Vlasov equation\cite{WessonD}, which uses the Lorentz force law to determine the time-evolution
of the distribution function for a species in a plasma,
   \begin{equation}
      \frac{\partial f_j}{\partial t} + \bm{v}_j\cdot\nabla f_j + \frac{q_j}{m_j}(\bm{E} + \bm{v_j}\times\bm{B})\cdot\nabla_v f_j = 0\;.
   \end{equation}
   %
It is trivial to show that a Maxwellian distribution satisfies (1.2.7). In the case of allowing collisions/interactions between species the
Vlasov equation is amended to the Fokker-Planck equation which contains on the right hand side terms for interactions between each pair of species\cite{WessonD},
   \begin{equation}
      \frac{\partial f_j}{\partial t} + \bm{v}_j\cdot\nabla f_j
      + \frac{q_j}{m_j}(\bm{E} + \bm{v_j}\times\bm{B})\cdot\nabla_v f_j = \left(\frac{\partial f_j}{\partial t}\right)_c.
   \end{equation}
   %
Note that when (1.2.9) is summed over all species, the pairs of collision terms will cancel due to Newton's $3^{rd}$ law,
giving again $0$ on the right-hand side.

\subsection{Tokamak Geometry}
   \quad For the tokamak geometry, we consider, for simplicity, an axisymmetric, large aspect-ratio, circular tokamak.
Using the Grad-Shafranov equation, the following definition for the equilibrium magnetic field can be found,
   \begin{equation}
      \bm{B} = B_\theta\bm{\hat{e}}_\theta + B_\zeta\bm{\hat{e}}_\zeta =
               B_\theta\bm{\hat{e}}_\theta + B_0(1-\epsilon\cos\theta)\bm{\hat{e}}_\zeta =
               B_0[\frac{\epsilon}{q}\bm{\hat{e}}_\theta + (1-\epsilon\cos\theta)\bm{\hat{e}}_\zeta],
   \end{equation}
   %
where $\epsilon=\frac{r}{R_0} \ll 1$ is the inverse aspect ratio, with $R=R_0+r\cos\theta$, for $r$ the minor radius,
and $R_0$ the major radius, and q $\simeq\frac{rB_\zeta}{R_0B_\theta}\sim1$ is the safety factor - the number of
toroidal turns required for one poloidal turn of magnetic field lines. The term $\epsilon\cos\theta$ in $R$ takes into
account the change in toroidal radius along the tokamak midplane. Working to $\mathcal{O}(\epsilon)$, the magnetic field
magnitude, magnetic field unit vector, and toroidal gradient terms can be written as,
   \begin{equation}
      B \equiv \sqrt{\bm{B}\cdot\bm{B}} = \sqrt{B_0^2[(1-\epsilon\cos\theta)^2 + (\frac{\epsilon}{q})\strut^2]} =
          B_0\sqrt{1-2\epsilon\cos\theta} \simeq B_0(1-\epsilon\cos\theta),
   \end{equation}
   %
   \begin{equation}
      \bm{\hat{b}} \equiv \frac{\bm{B}}{B} = \frac{\frac{\epsilon}{q}\bm{\hat{e}}_\theta + (1-\epsilon\cos\theta)\bm{\hat{e}}_\zeta}
                     {1-\epsilon\cos\theta} \simeq \frac{\epsilon}{q}(1+\epsilon\cos\theta)\bm{\hat{e}}_\theta + 
                     \bm{\hat{e}}_\zeta \simeq \frac{\epsilon}{q}\bm{\hat{e}}_\theta + \bm{\hat{e}}_\zeta,
   \end{equation}
   %
   \begin{equation}
      \nabla = \partial_r\bm{\hat{e}}_r + \frac{1}{r}\partial_\theta\bm{\hat{e}}_\theta +
               \frac{1}{R}\partial_\zeta\bm{\hat{e}}_\zeta = \partial_r\bm{\hat{e}}_r   +
               \frac{1}{r}\partial_\theta\bm{\hat{e}}_\theta + \frac{1}{R_0 + r\cos\theta}\partial_\zeta\bm{\hat{e}}_\zeta\;.
   \end{equation}


\section{Quasi-linear Theory}
   \quad The theoretical picture of turbulent transport is that the free energy released by an instability drives a steady level of
fluctuations in associated perturbed quantities, which results in radial transport of particles and energy. Precise relationships
between the fluctuations and the corresponding transport can be obtained by quasi-linear theory.\cite{WessonC}. In quasi-linear
theory, it is assumed that the plasma is weakly unstable, and that the instability leads to a broad spectrum of waves that modify
the background plasma in a self-consistent way via nonlinear interactions\cite{GurnBhatA}. Generally, quantities of interest are taken
to be a main spatially-averaged part which varies slowly with time compared to the frequency of perturbations, summed with a fluctuation
quantity. Within our plasma description, the density and velocity contributing to the flux are written as, 
   \begin{equation}
      n = \langle n \rangle + \delta n,
   \end{equation}
   %
   \begin{equation}
      \bm{v} = \delta \bm{v},
   \end{equation}
   %
where $\langle ... \rangle$ represents a flux-surface average, and $\delta \bm{v}$ represents velocity fluctuations. We are interested in
the effects of linear perturbations and so assume the form,   
   \begin{equation}
      A(x,t) = A_0e^{-i\omega t + i\bm{k}\cdot\bm{x}} 
   \end{equation}

\subsection{Anomalous Transport}
   \quad (2.0.1) and (2.0.2) together give a possibility for the anomalous turbulent transport, namely,
   \begin{equation}
      \Gamma_r = \langle\delta v_r \delta n\rangle_\theta,
   \end{equation}
   %
noting that we are considering flow in the radial direction, so averaging over the poloidal angle when considering the tokamak
geometry described in the following section. The lowest order term is second-order in perturbed quantities because
$\langle \langle n \rangle \delta v_r \rangle = \langle n \rangle \langle \delta v_r \rangle = 0$, since the average of an averaged
quantity is itself, and the average of truly random fluctuations is zero. The density perturbation can be written in terms of a
perturbed distribution function integrated over all velocity space, and the perturbed velocity as an $\bm{E}\times\bm{B}$ drift
where the perturbed $\bm{E}$ field and the background $\bm{B}$ field are used. Taking the poloidal average and bringing the drift
velocity into the velocity-space integral, we get the following result for the quasi-linear particle flux, 
   \begin{equation}
      \Gamma_r = \langle\int\delta f \frac{\delta\bm{E}\times\bm{\hat{b}}}{B} d^{3}v\rangle_\theta
               = \frac{1}{2\pi}\int_{0}^{2\pi} \int\delta f \frac{\delta E_\theta}{B}d\theta d^{3}v \;.
   \end{equation}
   %
Note there is a subtle point that only the $\theta$-component of $\delta E$ remains because we have dropped the term of
$\mathcal{O}(\epsilon)$ in the cross product with (1.9). For quasi-linear theory, equations will be linearized
and a Fourier transform can be taken to give a flux for each mode,
   \begin{equation}
      \Gamma_r = \frac{1}{2\pi}\int_{0}^{2\pi} \int\delta\widetilde{f}
                    \frac{(-\nabla_\theta(\delta\widetilde{\Phi}))}{B}d\theta d^{3}v
               = -\frac{ik_\theta}{2\pi B}\int_{0}^{2\pi}\int\delta\widetilde{f}
                    \delta\widetilde{\Phi}d\theta d^{3}v,
   \end{equation}
   %
where a $\sim$ over a quantity represents a Fourier mode. (2.1.3) can be made more general by allowing $\delta\widetilde{f}$ and $\delta\widetilde{\Phi}$
to be complex when calculating a real flux value. Given the following complex relations, the flux can be rewritten,
   \begin{equation}
      \operatorname{\mathbb{I}m}\{\delta\widetilde{f}\delta\widetilde{\Phi}\} = \operatorname{\mathbb{I}m}\{(a+ib)(c+id)\}
         = bc + ad,
   \end{equation}
   %
   \begin{equation}
   \begin{aligned}
      \delta\widetilde{f}^*\delta\widetilde{\Phi} + \delta\widetilde{f}\delta\widetilde{\Phi}^*
         &= [(a-ib)(c+id) + (a+ib)(c-id)]             \\
         &= [ac-ibc+iad+bd+ac+ibc-iad+bd] = 2[ac+bd], \\
      \delta\widetilde{f}^*\delta\widetilde{\Phi} - \delta\widetilde{f}\delta\widetilde{\Phi}^*
         &= [(a-ib)(c+id) - (a+ib)(c-id)]             \\
         &= [ac-ibc+iad+bd-ac-ibc+iad-bd] = 2i[ad-bc],
   \end{aligned}
   \end{equation}
   %
   \begin{equation}
      \Gamma_r = -\frac{ik_\theta}{2\pi B}\int_{0}^{2\pi}\int(\delta\widetilde{f}^*\delta\widetilde{\Phi}
                 - \delta\widetilde{f}\delta\widetilde{\Phi}^*)d\theta d^{3}v\;.
   \end{equation}


\section{Gyrokinetics}
   \quad Gyrokinetics provides a framework to study plasma behavior on perpendicular scales comparable to that of the
particle gyroradius $\rho$ and frequencies much lower than the particle cyclotron frequencies $\omega_c$. This model assumes
the following: $\rho \ll L$, with L the characteristic macroscopic plasma scale; $\omega \ll \omega_{c,i} \ll \omega_{ce}$;
and $k_\perp \rho_i \sim 1$\cite{GyroKinAstr}. The trajectory of these particles is decomposed into a slow guiding center motion
along the field line and a fast circular motion around the field line. The gyrokinetic Vlasov equation is found by converting the
standard Vlasov equation to guiding center coordinates $\bm{X}$ and $\bm{V}$ and averaging over the gyrophase angle $\alpha
\equiv \omega_c t$, to account for the fast circular motion, where\cite{FriemanChen},
   \begin{equation}
      \bm{X} = \bm{x} + \frac{\bm{v}\times\bm{\hat{b}}}{\omega_c} = \bm{x} + \frac{\bm{v}_\perp}{\omega_c}
             = \bm{x} + \bm{\rho},
   \end{equation}
   %
   \begin{equation}
      \bm{V} = \bm{V}(\epsilon^*, \mu, \alpha) = \dot{\bm{X}},
   \end{equation}
   %
where $\epsilon^*$ is defined in section 1.2, $\mu \equiv \frac{v_\perp^2}{2B}$, and $\bm{v}_\perp =
v_\perp[\cos\alpha\bm{\hat{e}}_1 + \sin\alpha\bm{\hat{e}}_2]$\cite{FriemanChen}, with $\bm{\hat{e}}_1$
and $\bm{\hat{e}}_2$ being local orthogonal unit vectors. We now note that the gradient defined
in (1.3.4) is the gradient in guiding center coordinates because the geometry of the field lines
is that of the guiding center - i.e. $\nabla = \nabla_X$ from now on.

\subsection{Gyrokinetic Vlasov Equation}
   \quad The perturbed, gyrokinetic distribution function is given as a combination of adiabatic
and non-adiabatic terms\cite{FriemanChen},
   \begin{equation}
      \delta F = \frac{q}{m}\delta F_a + \delta G,
   \end{equation}
   %
where,    
   \begin{equation}
      \delta F_a = \delta\Phi[\frac{\partial}{\partial\epsilon^*} + \frac{\partial}{B\partial\mu}]F_0, 
   \end{equation}
   %
   \begin{equation}
      \delta G_0 = -\frac{q}{m}\langle\delta\Phi\rangle_\alpha\frac{\partial F_0}{B\partial\mu} + \delta H_0,
   \end{equation}
   %
   \begin{equation}
      \langle\ldots\rangle_\alpha = \frac{1}{2\pi}\int_{0}^{2\pi}(\ldots)d\alpha\;.
   \end{equation}
   %
Higher order terms in $\delta G$ have been dropped, as well as magnetic fluctuation terms - i.e. we choose to
work in the electrostatic limit. We can simplify things further by choosing for $F_0$ a Maxwellian equilibrium
distribution function, $f_M$, so that it only depends on $\epsilon^*$ and not $\mu$ or $\alpha$. This gives us
a final distribution function,
   \begin{equation}
      \delta F = \frac{q}{m}\delta\Phi\frac{\partial}{\partial\epsilon^*}f_M + \delta H_0\;.
   \end{equation}
   %
This distribution function can be plugged into the Vlasov equation and gyrophase-averaged to give the standard gyrokinetic Vlasov
equation for a species j\cite{FriemanChen},
   \begin{equation}
   \begin{aligned}
        \partial_t\delta H_0 + v_\parallel\nabla_{X_\parallel}\delta H_0 +
        (\bm{v}_d + \frac{\bm{\hat{b}}\times\nabla_X\langle\delta\Phi\rangle_\alpha}{B})\cdot\nabla_X\delta H_0 \\
      = -\frac{q_j}{m_j}[\partial_t\langle\delta\Phi\rangle_\alpha\partial_{\epsilon^*} f_M
        -\frac{1}{\omega_{c,j}}(\nabla_X\langle\delta\Phi\rangle_\alpha\times\bm{\hat{b}})\cdot\nabla_X f_M],
   \end{aligned}
   \end{equation}
   %
where $\bm{v}_d$, the sum of magnetic curvature and gradient drift terms, is defined as,
   \begin{equation}
   \begin{aligned}
      \bm{v}_d = \frac{v_\parallel^2 + \frac{1}{2} v_\perp^2}{\omega_{c,j}}\frac{\bm{B}\times\nabla B}{B^2},
   \end{aligned}
   \end{equation}
   %
with, simplifying to lowest order in $\epsilon$,
   \begin{equation}
   \begin{aligned}
      \frac{\bm{B}\times\nabla B}{B^2} &= \frac{B_0[(1-\epsilon\cos\theta)\bm{\hat{e}}_\zeta + \frac{\epsilon}{q}\bm{\hat{e}}_\theta]
      \times(\partial_r\bm{\hat{e}}_r + \frac{1}{r}\partial_\theta\bm{\hat{e}}_\theta + \frac{1}{R}\partial_\zeta\bm{\hat{e}}_\zeta)
      B_0(1-\epsilon\cos\theta)}{B_0^2(1-\epsilon\cos\theta)^2} \\ &=
      \frac{[1-\epsilon\cos\theta)\bm{\hat{e}}_\zeta + \frac{\epsilon}{q}\bm{\hat{e}}_\theta]
      \times[-\frac{1}{R_0}\cos\theta\bm{\hat{e}}_r + \frac{r}{r R_0}\sin\theta\bm{\hat{e}}_\theta]}{(1-\epsilon\cos\theta)^2} \\ &
      \begin{aligned}
         \;= \frac{1}{(1-\epsilon\cos\theta)^2}[&-\frac{(1-\epsilon\cos\theta)\cos\theta}{R_0}(\bm{\hat{e}}_\zeta\times\bm{\hat{e}}_r)
                                                 -\frac{(1-\epsilon\cos\theta)\cos\theta}{R_0}(\bm{\hat{e}}_\zeta\times\bm{\hat{e}}_\theta) \\
                                                &-\frac{\epsilon}{q R_0}\cos\theta(\bm{\hat{e}}_\theta\times\bm{\hat{e}}_r)]
      \end{aligned}         
      \\ &\simeq (1+2\epsilon\cos\theta)[-\frac{\cos\theta}{R_0}\bm{\hat{e}}_\theta-\frac{\sin\theta}{R_0}\bm{\hat{e}}_r]
          \simeq -\frac{1}{R_0}(\sin\theta\bm{\hat{e}}_r + \cos\theta\bm{\hat{e}}_\theta)\;.
   \end{aligned}
   \end{equation}
   %
The second and third terms on the left-hand side of (3.1.6) can be simplified to lowest order in $\epsilon$ using (1.3.1)-(1.3.4) and
(3.1.7)-(3.1.8) as,
   \begin{equation}
   \begin{aligned}
         v_\parallel\nabla_{X_\parallel} &= 
         v_\parallel(\bm{\hat{b}}\cdot\nabla) = v_\parallel(\frac{\epsilon}{q}\bm{\hat{e}}_\theta +
         \bm{\hat{e}}_\zeta)\cdot(\partial_r\bm{\hat{e}}_r + \frac{1}{r}\partial_\theta\bm{\hat{e}}_\theta + \frac{1}{R_0 +
         r\cos\theta}\partial_\zeta\bm{\hat{e}}_\zeta) \\ &= 
         v_\parallel(\frac{\epsilon}{q r}\partial_\theta + \frac{1}{R}\partial_\zeta) = v_\parallel(\frac{1}{q R_0}\partial_\theta
         + \frac{1}{R}\partial_\zeta) = \frac{v_\parallel}{q R}(\frac{R}{R_0}\partial_\theta + q\partial_\zeta) \\ &=
         \frac{v_\parallel}{q R}((1+\epsilon\cos\theta)\partial_\theta + q\partial_\zeta)
         \simeq \frac{v_\parallel}{q R}(\partial_\theta + q\partial_\zeta) = v_\parallel\frac{\partial}{\partial l},
   \end{aligned}
   \end{equation}
   %
   \begin{equation}
   \begin{aligned}
      \bm{v}_d\cdot\nabla_X = \bm{v}_d\cdot\nabla &= -\frac{v_\parallel^2 + \frac{1}{2}v_\perp^2}{\omega_{c,j}}
      (\sin\theta\bm{\hat{e}}_r + \cos\theta\bm{\hat{e}}_\theta)
      \cdot(\partial_r\bm{\hat{e}}_r + \frac{1}{r}\partial_\theta\bm{\hat{e}}_\theta) \\
      &= -\frac{v_\parallel^2 + \frac{1}{2}v_\perp^2}{\omega_{c,j}R_0}(\sin\theta\partial_r + \frac{\cos\theta}{r}\partial_\theta),
   \end{aligned}
   \end{equation}
   %
with $l$ being the length along the field lines. Note that we have dropped the non-linear $\bm{E}\times\bm{B}$ drift term on the left-hand
side of (3.1.6) because we are interested in linearizing this equation.

   \quad Since the gyrokinetic analysis employs two spatial scales, Frieman and Chen rewrite functions of $\bm{x}$ in terms of $\bm{X}$ using
a WKB approximation\cite{FriemanChen},
   \begin{equation}
      A(\bm{x}, \bm{v}) = \sum\limits_{\bm{k}_\perp}\bar{A}(\bm{x},\bm{v},k_\perp)e^{i\int_{\bm{x}_{\perp 0}}\bm{k}\cdot d\bm{x}_\perp}\;.
   \end{equation}
   %
where $\bar{A}$ and $\bm{k}_\perp$ contain slow spatial variations. Ignoring nonlinear terms and dropping the sum, then integrating-by-parts
with the definition from (3.0.1), we can rewrite (3.1.11) as\cite{FriemanChen},
   \begin{equation}
   \begin{aligned}
      A(\bm{x}, \bm{v}, k_\perp) &= \bar{A}(\bm{X},\bm{V},k_\perp)e^{i(\int_{\bm{X}_{\perp 0}}\bm{k}_\perp\cdot\bm{X} - \bm{k}_\perp\cdot\bm{\rho})} \\
                                 &= A(\bm{X},\bm{V},k_\perp)e^{-i \bm{k}_\perp\cdot\bm{\rho}}.
   \end{aligned}                              
   \end{equation}
   %
Note that for macroscopic quantities $\rho/L \ll 1$, meaning $A(\bm{x}) \simeq A(\bm{X})$.  It is also worth noting that the choice of direction of
$\bm{k}_\perp$ is arbitrary when gyrophase-averaging. To see this, we can choose an arbitrary angle, $\phi$, for $\bm{k}_\perp$ to give,
   \begin{equation}
      e^{-i\bm{k}_\perp\cdot\bm{\rho}} = e^{-i\frac{k_\perp v_\perp}{\omega_c}\cos(\alpha - \phi)} = e^{-iz\cos(\alpha-\phi)},
   \end{equation}
   %
where $z = \frac{k_\perp v_\perp}{\omega_c}$. Then the gyrophase-average gives a final result independent of $\phi$,
   \begin{equation}
      \frac{1}{2\pi}\int_{0}^{2\pi}e^{-iz\cos(\alpha-\phi)}d\alpha = J_0(z)\;.
   \end{equation}
   %
Because of this arbitariness, the standard choice is to pick $\bm{k}_\perp$ in the direction $\bm{\hat{e}}_2$, as defined in
$\bm{v}_\perp$ in the gyrokinetic introduction above. This leads to the following form
for gyrophase-averages,
   \begin{equation}
   \begin{aligned}
      \langle A(\bm{x})\rangle_\alpha &= \frac{1}{2\pi}\int_{0}^{2\pi}\bar{A}(\bm{X},\bm{V},k_\perp)e^{-iz\sin\alpha} d\alpha \\
                                      &= J_0(z) A(\bm{X}) = J_0(z)A(\bm{X}) = J_0(z)A(\bm{X}),
   \end{aligned}
   \end{equation}
   %
with the following definition for the zeroth-order Bessel function,
   \begin{equation}
      J_0(z) = \frac{1}{2\pi}\int_{0}^{2\pi}e^{\pm iz\sin\alpha}d\alpha\;.
   \end{equation}

\subsection{Gyrokinetic Distribution Function}
   \quad We can simplify (3.1.6) further by plugging in (3.1.9), (3.1.10), and (3.1.15) and then taking the Fourier transform, giving,
   \begin{equation}
   \begin{aligned}
      (-i\omega + v_\parallel\frac{\partial}{\partial l} + i \bm{v}_d\cdot\bm{k}_X)\delta\widetilde{H}_0 &=
      -\frac{q_j}{m_j}[-i\omega J_0(z_j)\delta\widetilde{\Phi}\partial_{\epsilon^*}f_{M,j} \\
      -\frac{i}{\omega_{c,j}}J_0(z_j)\delta\widetilde{\Phi}(\bm{k}\times\bm{\hat{b}})\cdot\frac{d}{dr} f_{M,j}\bm{\hat{e}}_r] &=
      i\frac{q_j}{m_j}J_0(z_j)\delta\widetilde{\Phi}[\omega\partial_{\epsilon^*} + \frac{k_\theta}{\omega_{c,j}}\frac{d}{dr}]f_{M,j},
   \end{aligned}
   \end{equation}
   %
   \begin{equation}
      \Rightarrow\; (v_\parallel\partial_l - i(\omega - \bar{\omega}_{dj}))\delta\widetilde{H}_0 =
      i\frac{q_j}{m_j}J_0(z_j)\delta\widetilde{\Phi}[\omega\partial_{\epsilon^*} + \frac{k_\theta}{\omega_{c,j}}\frac{d}{dr}]f_{M,j},
   \end{equation}
   %
with the following definitions,
   \begin{equation}
   \begin{aligned}
      \bar{\omega}_{dj} &\equiv -\bm{v}_d\cdot\bm{k} = \frac{k_\theta(v_\parallel^2 + \frac{1}{2}v_\perp^2)}{\omega_{c,j}R_0}(\cos\theta
                                + \frac{k_r}{k_\theta}\sin\theta) \\
                        &= \frac{\omega_{d,j}}{2}(\left(\frac{v_\parallel}{v_{T,j}}\right)^2 + \frac{1}{2}\left(\frac{v_\perp}{v_{T,j}}\right)^2)
                           (\cos\theta + \hat{s}\theta\sin\theta),  
   \end{aligned}
   \end{equation}
   %
   \begin{equation}
      \bm{k}\times\bm{\hat{b}} = (k_r\bm{\hat{e}}_r + k_\theta\bm{\hat{e}}_\theta + k_\zeta\bm{\hat{e}}_\zeta)\times
      (\frac{\epsilon}{q}\bm{\hat{e}}_\theta + \bm{\hat{e}}_\zeta) \simeq -k_r\bm{\hat{e}}_\theta + k_\theta\bm{\hat{e}}_r,
   \end{equation}
   %
where $\partial_r = i k_\theta\hat{s}\theta \Rightarrow k_r = k_\theta\hat{s}\theta$, for magnetic shear $\hat{s}$\cite{HC}. $\omega_{d,j}$ represents
the magnetic curvature drift frequency,
   \begin{equation}
      \omega_{d,j} \equiv 2\frac{n}{dn/dr}\frac{\omega_{*,j}}{R_0} = 2\frac{k_\theta T_j}{q_j B R_0},
   \end{equation}
   %
and $\omega_{*,j}$ the diamagnetic drift frequency,
   \begin{equation}
      \omega_{*,j} \equiv \frac{k_\theta T_j}{q_j B}\frac{1}{n}\frac{dn}{dr}\;.
   \end{equation}
   %
Note that $f_{M,j}$ depends only on r due to it being a function of $n_j(r)$ and $T_j(r)$ which are only changing across the circular
flux-surfaces - therefore only functions of radius. Next, we can simplify further by plugging in the definition of the Maxwellian,
giving the following values for derivatives,
   \begin{equation}
      \partial_{\epsilon^*}f_{M,j} = -\frac{m_j}{T_j}n_j\left(\frac{m_j}{2\pi T_j}\right)^{3/2}e^{-\frac{m_j \epsilon^*}{T_j(r)}} =
                                 -\frac{m_j}{T_j}f_{M,j},
   \end{equation}
   %
   \begin{equation}
   \begin{aligned}
      \frac{d}{dr}f_{M,j} &= \frac{dn}{dr}\frac{d}{dn}f_{M,j} + \frac{dT}{dr}\frac{d}{dT}f_{M,j} = 
                         \frac{dn}{dr}e^{-\frac{m_j \epsilon^*}{T_j(r)}} + 
                         \frac{dT}{dr}\frac{du}{dT}\frac{d}{du}[n_j\left(\frac{m_j u(T(r))}{2\pi}\right)^{3/2}e^{-m_j \epsilon^* u(T_j(r))}] \\
                      &= \frac{1}{n}\frac{dn}{dr}f_{M,j} + \frac{dT}{dr}\frac{du}{dT}n_j[\frac{3}{2}\left(\frac{m_ju}{2\pi}\right)^{1/2}
                         \left(\frac{m_j}{2\pi}\right) - m_j\epsilon^*\left(\frac{m_ju}{2\pi}\right)^{3/2}]e^{-m_j \epsilon^* u(T_j(r))} \\
                      &= \frac{1}{n}\frac{dn}{dr}f_{M,j} + \frac{dT}{dr}\frac{du}{dT}[\frac{3}{2}u^{-1} - m_j\epsilon^*]f_{M,j}
                       = [\frac{1}{n}\frac{dn}{dr} + \frac{dT}{dr}(-\frac{1}{T^2})(\frac{3}{2}u^{-1} - m_j\epsilon^*)]f_{M,j} \\
                      &= [\frac{1}{n}\frac{dn}{dr} - \frac{1}{T}\frac{dT}{dr}(\frac{3}{2} - \frac{m_jv^2}{2T_j})]f_{M,j}
                       = [\frac{1}{n}\frac{dn}{dr} - \frac{1}{T}\frac{dT}{dr}(\frac{3}{2} - \frac{v^2}{2v_{T,j}^2})]f_{M,j} \\
                      &= \frac{1}{n}\frac{dn}{dr}[1 + (\frac{1}{2}\left(\frac{v}{v_{T,j}}\right)^2 - \frac{3}{2})\eta_j]f_{M,j},
   \end{aligned}
   \end{equation}
   %
given that $u = T^{-1}$, $du = -T^{-2}dT$, $v_{T,j} \equiv \sqrt{\frac{T_j}{m_j}}$, and $\eta_j \equiv \frac{n}{T}\frac{dT}{dn}$. Putting these
derivatives into (3.2.2) then gives a nice form for the gyrokinetic equation,
   \begin{equation}
   \begin{aligned}
      &(v_\parallel\partial_l - i(\omega - \bar{\omega}_{dj}))\delta\widetilde{H}_0 =               \\ &-i\frac{q_j}{m_j}J_0(z_j)\delta\widetilde{\Phi}
          [-\frac{m_j}{T_j}\omega + \frac{k_\theta}{\omega_{c,j}}\frac{1}{n}\frac{dn}{dr}[1 + (\frac{1}{2}\left(\frac{v}{v_{T,j}}\right)^2 - \frac{3}{2})\eta_j]]f_{M,j} \\
      \Rightarrow\; &(v_\parallel\partial_l - i(\omega - \bar{\omega}_{dj}))\delta\widetilde{H}_0 = \\ &-i\frac{q_j}{T_j}J_0(z_j)\delta\widetilde{\Phi}
          [\omega - \frac{k_\theta T_j m_j}{q_j B m_j}\frac{1}{n}\frac{dn}{dr}[1 + (\frac{1}{2}\left(\frac{v}{v_{T,j}}\right)^2 - \frac{3}{2})\eta_j]]f_{M,j}           \\
      \Rightarrow\; &(-v_\parallel\partial_l + i(\omega - \bar{\omega}_{dj}))\delta\widetilde{H}_0 = \\ &i\frac{q_j}{T_j}J_0(z_j)\delta\widetilde{\Phi}
          [\omega - \omega_{*,j}[1 + (\frac{1}{2}\left(\frac{v}{v_{T,j}}\right)^2- \frac{3}{2})\eta_j]]f_{M,j}
   \end{aligned}
   \end{equation}
   %
   \begin{equation}
      \Rightarrow\; (-v_\parallel\partial_l + i(\omega - \bar{\omega}_{dj}))\delta\widetilde{H}_0
      = i\frac{q_j}{T_j}J_0(z_j)\delta\widetilde{\Phi}[\omega - \omega_{*,j}^T]f_{M,j},
   \end{equation}
   %
with the following definition,
   \begin{equation}
      \omega_{*,j}^T \equiv \omega_{*,j}[1 + (\frac{1}{2}\left(\frac{v}{v_{T,j}}\right)^2-\frac{3}{2})\eta_j]\;.
   \end{equation}
   %

\subsection{Physical Approximations}
   \quad (3.2.10) can be simplified further and solved for $\delta\widetilde{H}_0$ if a few physical assumptions are made for electron drift and ITG modes.
First off, trapped particle effects are neglected, i.e.,
   \begin{equation}
      \omega \gg \omega_{bj},
   \end{equation}
   %
for $\omega_{bj}$ the bounce frequency. We also assume electrons move rapidly in response to the electrostatic potential,
   \begin{equation}
      k_\parallel v_{T,e} \gg \omega \gg k_\parallel v_{T,i},
   \end{equation}
   %
noting that $v_{T,j} \simeq v_{\parallel,j}$, since the perpendicular velocities come from the smaller drift terms. Finally we assume that the frequency
associated with magnetic drifts of ions is much smaller than that of the perturbed modes,
   \begin{equation}
      \omega \gg \omega_{di}\;.
   \end{equation}
   %
Now, (3.3.2) and (3.3.3) allow the left-hand side of (3.2.10) to be rewritten for ions as,
   \begin{equation}
      i(\omega-\bar{\omega}_{di})(1+i\frac{v_\parallel}{\omega-\omega_{di}}\frac{\partial}{\partial l}) \simeq
      i(\omega-\bar{\omega}_{di})(1-\frac{k_\parallel v_{\parallel i}}{\omega}) \simeq i(\omega-\bar{\omega}_{di})\;.
   \end{equation}
   %
Finally, we can solve for $\delta\widetilde{H}_{0,i}$ explicitly by replacing the left-hand side of (3.2.10) with (3.3.4), giving
the perturbed, non-adiabatic distribution function for ions,
   \begin{equation}
      \delta\widetilde{H}_{0,i} = \frac{q_i}{T_i}J_0(z_i)\delta\widetilde{\Phi}
                                      f_{M,i}\frac{\omega-\omega_{*i}^T}{\omega-\bar{\omega}_{di}},
   \end{equation}
   %
or, converting back to $\bm{x}$ coordinates using (3.1.12) for the velocity-space integration to get $n_j$,
   \begin{equation}
      \delta\widetilde{H}_{0,i}(\bm{x}) = \frac{q_i}{T_i}J_0(z_i)\delta\widetilde{\Phi}(\bm{x})
                                      f_{M,i}\frac{\omega-\omega_{*i}^T}{\omega-\bar{\omega}_{di}}e^{iz\sin\alpha}\;.
   \end{equation}
   %
Note that again the arbitrary choice in the direction of $\bm{k}_\perp$ is invokable because $\alpha$ must be
integrated over again when the distribution function is integrated over velocity-space to give the flux. We continue
to treat electrons as adiabatic for now.

\subsection{Quasi-linear Flux}
   \quad We can now plug our perturbed distribution functions into the quasi-linear estimate for radial flux, (2.1.6). The total
distribution functions are now given as,
   \begin{equation}
      \delta \widetilde{f}_j = -\frac{q_j}{T_j}\delta\widetilde{\Phi}f_{M,j} + \delta\widetilde{H}_{0,j},
   \end{equation}
   %
where $\delta H_0 = 0$ for adiabatic electrons. Plugging into (2.1.6) then gives,
   \begin{equation}
   \begin{aligned}
      \Gamma_{r,j} = -\frac{ik_\theta}{2\pi B}\int_{0}^{2\pi}d\theta\int d^3v
            [&\frac{q_j}{T_j}\delta\widetilde{\Phi}\delta\widetilde{\Phi}^*f_{M,j}(1+J_0(z_j)\frac{\omega^*-\omega^T_{*,j}}{\omega^*-\bar{\omega}_{d,j}}e^{-iz\sin\alpha}) \\
             & + \frac{q_j}{T_j}\delta\widetilde{\Phi}\delta\widetilde{\Phi}^*f_{M,j}(1+J_0(z_j)\frac{\omega-\omega^T_{*,j}}{\omega-\bar{\omega}_{d,j}}e^{iz\sin\alpha})]\;.
   \end{aligned}
   \end{equation}
   %
Ignoring the adiabatic perturbation because it can't lead to a real flux due to the $i$ out front and simplifying the integral over $\alpha$ with
(3.1.16) then gives,
   \begin{equation}
   \begin{aligned}
      \Gamma_{r,j} &= -\frac{ik_\theta}{B}\frac{q_j}{T_j}\lvert\delta\widetilde{\Phi}\rvert^2\int_{0}^{2\pi}d\theta\int_{-\infty}^{\infty}dv_\parallel
                      \int_{0}^{\infty}v_\perp dv_\perp J_0^2(\frac{k_\perp v_\perp}{\omega_{c,j}})
                      [\frac{\omega^*-\omega^T_{*,j}}{\omega^*-\bar{\omega}_{d,j}} + \frac{\omega-\omega^T_{*,j}}{\omega-\bar{\omega}_{d,j}}]f_{M,j} \\
                   & 
                      \begin{aligned} \;=
                      -\frac{ik_\theta}{B}\frac{q_j}{T_j}\lvert\delta\widetilde{\Phi}\rvert^2\left(\frac{m_j}{2\pi T_j}\right)^{3/2}
                      \int_{0}^{2\pi}d\theta\int_{-\infty}^{\infty}dv_\parallel e^{-\frac{v_\parallel^2}{v_{T,j}^2}}
                      \int_{0}^{\infty}v_\perp dv_\perp e^{-\frac{v_\perp^2}{v_{T,j}^2}}
                      J_0^2(\frac{k_\perp v_\perp}{\omega_{c,j}})\times \\
                      [\frac{\omega^*-\omega^T_{*,j}}{\omega^*-\bar{\omega}_{d,j}}
                       + \frac{\omega-\omega^T_{*,j}}{\omega-\bar{\omega}_{d,j}}]
                      \end{aligned}
   \end{aligned}
   \end{equation}

\subsection{Quasi-neutrality Condition}
   \quad Taking adiabatic electrons and kinetic ions with the quasi-neutrality condition (1.2.5) gives,
   \begin{equation}
      q_e n_e + q_i n_i = 0 \Rightarrow -e n_e + e n_i = 0 \Rightarrow n_e = n_i
   \end{equation}
   %
   \begin{equation}
   \begin{aligned}
      \Rightarrow \int(f_{M,e}-\frac{q_e}{T_e}\delta\widetilde{\Phi}f_{M,e})d^3v
                 &= \int(f_{M,i} - \frac{q_i}{T_i}\delta\widetilde{\Phi}f_{M,i} + \delta\widetilde{H}_{0,i})d^3v \\
      \Rightarrow n_0 + \frac{e}{T_e}n_0\delta\widetilde{\Phi}
                 &= n_0 - \frac{e}{T_i}n_0\delta\widetilde{\Phi} + \int\delta\widetilde{H}_{0,i}d^3v
   \end{aligned}
   \end{equation}
   %
   \begin{equation}
   \begin{aligned}
      \Rightarrow  en_0\delta\widetilde{\Phi}(\frac{1}{T_e} + \frac{1}{T_i})
                &= \frac{e}{T_i}\delta\widetilde{\Phi}\int_{-\infty}^{\infty}dv_\parallel\int_{0}^{\infty}v_\perp dv_\perp
                   J_0(\frac{k_\perp v_\perp}{\omega_{c,i}})\frac{\omega - \omega_{*i}^T}{\omega - \bar{\omega}_{di}}f_{M,i}\int_{0}^{2\pi}e^{-iz\sin\alpha}d\alpha \\
      \Rightarrow n_0(\frac{1}{T_e} + \frac{1}{T_i}) &= \frac{2\pi}{T_i}\int_{-\infty}^{\infty}dv_\parallel\int_{0}^{\infty}v_\perp dv_\perp
                  J_0^2(\frac{k_\perp v_\perp}{\omega_{c,i}})\frac{\omega - \omega_{*i}^T}{\omega - \bar{\omega}_{di}}f_{M,i} \\
      \Rightarrow n_0(\frac{T_i}{T_e} + 1) &= \frac{1}{\sqrt{2\pi}}\left(\frac{m_i}{T_i}\right)^{3/2}\int_{-\infty}^{\infty}dv_\parallel e^{-\frac{v_\parallel^2}{2v^2_{T,i}}}
                  \int_{0}^{\infty}v_\perp dv_\perp J_0^2(\frac{k_\perp v_\perp}{\omega_{c,i}})\frac{\omega - \omega_{*i}^T}{\omega - \bar{\omega}_{di}}e^{-\frac{v_\perp^2}{2v^2_{T,i}}}\;.
   \end{aligned}
   \end{equation}


\section{Comparison to Simulation}
   Information about simulations.

\subsection{Impurities}
   \quad A third species, the impurity term, can easily be added to the above theory. The radial flux for the impurity ion is calculated the same way as for
the main ion species, and the basic quasi-neutrality condition has to be changed to the following,
   \begin{equation}
      n_e = n_i + Z_In_I,
   \end{equation}
   %
with $Z_I$ the charge of the impurity. (4.1.1) leads to an updated form of (3.5.2) with an impurity contribution,
   \begin{equation}
   \begin{aligned}
      \Rightarrow n_e + \frac{e}{T_e}n_e\delta\widetilde{\Phi} = n_i - \frac{e}{T_i}n_i\delta\widetilde{\Phi} + \int\delta\widetilde{H}_{0,i}d^3v
                      + Z_In_I - \frac{Z_Ie}{T_I}n_I\delta\widetilde{\Phi} + \int\delta\widetilde{H}_{0,I}d^3v\;.
   \end{aligned}
   \end{equation}
   %
The lowest order terms will cancel as before, giving an updated version of (3.5.3),
   \begin{equation}
      n_0(\frac{1}{T_e} + \frac{1}{T_i} + \frac{Z_I}{T_I}) = \frac{1}{\sqrt{2\pi}}\sum\limits_{ions}Z_j\left(\frac{m_j}{T_j}\right)^{3/2}
         \int_{-\infty}^{\infty}dv_\parallel e^{-\frac{v_\parallel^2}{2v^2_{T,j}}}
         \int_{0}^{\infty}v_\perp dv_\perp J_0^2(\frac{k_\perp v_\perp}{\omega_{c,j}})
         \frac{\omega - \omega_{*,j}^T}{\omega - \bar{\omega}_{dj}}e^{-\frac{v_\perp^2}{2v_{T,j}^2}}\;.
   \end{equation} 

\subsection{Normalization}


\begin{thebibliography}{0}
   
   \bibitem{ITER} ITER Physics Expert Groups on Confinement and Transport and Confine-ment Modelling and Database, Nucl. Fusion \textbf{39}, 2175 (1999).
   \bibitem{WessonA} Wesson J., \& Campbell D. J., \textit{Tokamaks} (Oxford: Clarendon Press, 2004), p.219
   \bibitem{WessonB} Wesson J., \& Campbell D. J., \textit{Tokamaks} (Oxford: Clarendon Press, 2004), p.196-197
   \bibitem{FriemanChen} Frieman E. A., Chen L., The Physics of Fluids \textbf{25}, 502 (1982).
   \bibitem{WessonC} Wesson J., \& Campbell D. J., \textit{Tokamaks} (Oxford: Clarendon Press, 2004), p.202-210
   \bibitem{GurnBhatA} Gurnett, D., \& Bhattacharjee, A., \textit{Introduction to Plasma Physics: With Space and Laboratory Applications},
                       (Cambridge: Cambridge University Press 2005), p.428
   \bibitem{GyroKinAstr} G.G. Howes, S.C. Cowley, W. Dorland, G.W. Hammett, E. Quataert, and A.A. Schekochihin.
                         Astrophysical gyrokinetics: Basic equations and linear theory. ApJ, \textbf{651}(1):590, (2006).
   \bibitem{HC} See Haotian...
   \bibitem{WessonD} Wesson J., \& Campbell D. J., \textit{Tokamaks} (Oxford: Clarendon Press, 2004), p.55-56
   %\bibitem{HazMeiss} R. D. Hazeltine, J. D. Meiss \textit{Plasma Confinement} (Dover Publications, 2003), p.36-43

\end{thebibliography}
    
\end{document}